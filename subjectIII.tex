\documentclass{article}

\usepackage[margin=1in]{geometry}

\usepackage{lmodern}
\usepackage{amsmath}
\usepackage{amssymb}
\usepackage{upgreek}
\usepackage{pgfplots}
\usepackage{qtree}
\usepackage{calc}
\usepackage[normalem]{ulem}
\usepackage[utf8]{inputenc}
\usepackage[T1]{fontenc}

\pgfplotsset{compat=1.11}
\usetikzlibrary{arrows, decorations.markings}
\usetikzlibrary{3d}
\usetikzlibrary{shapes.geometric,decorations.fractals,shadows}
\usepgfplotslibrary{polar}


\begin{document}

\section{Derivada Direcional e Vetor Gradiente}
\begin{tabbing}
Seja: \= $f : \mathbb{R} \times \mathbb{R} \rightarrow \mathbb{R}$. \\[5pt]
  \> $P = (a, b)$ um ponto no domínio de $f$. \\[5pt]
  \> $\vec{v} = (v_x, v_y)$ um vetor unitário. \\[5pt]
  \> $r(t) = (a + v_x \cdot t, \enspace b + v_y \cdot t)$ uma reta direcionada por $\vec{v}$ que passa pelo ponto $P$.
\end{tabbing}
\vspace{10pt}
A derivada de $f$ no ponto $P$ na direção de $\vec{v}$ é
\[ f_{\vec{v}}(a,b) = f_x(a,b) \cdot v_x + f_y(a,b) \cdot v_y \] \\
O vetor gradiente de $f$ no ponto $P$ é
\[ \nabla f(a,b) = f_x(a,b) \cdot \vec{i} + f_y(a,b) \cdot \vec{j} \]
Portanto, vale que $f_{\vec{v}}(a,b) = \langle \nabla f(a,b), \enspace \vec{v} \, \rangle$. \\[20pt]
Teorema: A direção em que $f$ tem a maior taxa de variação é a do vetor gradiente.

\end{document}
