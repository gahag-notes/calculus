\documentclass{article}

\usepackage[margin=1in]{geometry}

\usepackage{lmodern}
\usepackage{amsmath}
\usepackage{amssymb}
\usepackage{upgreek}
\usepackage{pgfplots}
\usepackage{qtree}
\usepackage{calc}
\usepackage[normalem]{ulem}
\usepackage[utf8]{inputenc}
\usepackage[T1]{fontenc}

\pgfplotsset{compat=1.11}
\usetikzlibrary{arrows, decorations.markings}
\usetikzlibrary{3d}
\usetikzlibrary{shapes.geometric,decorations.fractals,shadows}
\usepgfplotslibrary{polar}


\begin{document}


\section{Derivada Direcional e Vetor Gradiente}
\begin{tabbing}
Seja: \= $f : \mathbb{R} \times \mathbb{R} \rightarrow \mathbb{R}$. \\[5pt]
  \> $P = (a, b)$ um ponto no domínio de $f$. \\[5pt]
  \> $\vec{v} = (v_x, v_y)$ um vetor \uline{unitário}.
\end{tabbing}
\vspace{10pt}
A derivada de $f$ no ponto $P$ na direção de $\vec{v}$ é
\[ f_{\vec{v}}(a,b) = f_x(a,b) \cdot v_x + f_y(a,b) \cdot v_y \] \\
O vetor gradiente de $f$ no ponto $P$ é
\[ \nabla f(a,b) = f_x(a,b) \cdot \vec{i} + f_y(a,b) \cdot \vec{j} \]
Portanto, vale que $f_{\vec{v}}(a,b) = \langle \nabla f(a,b), \enspace \vec{v} \, \rangle$. Notando que $\vec{v}$ é unitário, então
\[ \langle \nabla f(a,b), \enspace \vec{v} \, \rangle \enspace = \enspace  ||\nabla f(a,b)|| \cdot  ||\vec{v}|| \cdot \cos \theta \enspace = \enspace  ||\nabla f(a,b)|| \cdot \cos \theta \] \\
Então, quando $\theta = 0$, a derivada direcional é a maior possível, e quando $\theta = \pi$ a derivada direcional é a menor possível. \\[20pt]
\uline{Teorema}: A direção em que $f$ tem a maior taxa de variação é a do vetor gradiente.



\section{Pontos Críticos e Pontos Extremais Locais}
Dizemos que $P = (a,b)$ é um ponto máximo ou mínimo (extremal) \uline{local} se existir um disco ao redor de $P$ tal que para todo ponto $(x,y)$ no disco, vale
\begin{align*}
  f(x,y) \leq f(a,b) & \text{\qquad(máximo)} \\
  f(x,y) \geq f(a,b) & \text{\qquad(mínimo)}
\end{align*}
Se $P$ é extremal, então $\nabla f(a,b) = \vec{0}$, e portanto $f_{\vec{v}}(a,b) = 0$. \\[10pt]
Dizemos que $P$ é um ponto crítico de $f$ se $\nabla f(a,b) = \vec{0}$. \\[10pt]
Obs.: Um ponto crítico pode \uline{não} ser extremal.

\subsection{Discriminante}
Se $P$ é um ponto crítico de $f$, então o discriminante de $f$ no ponto $P$ é
\[
  D(a,b) = \Bigg | \begin{bmatrix}
                    f_{xx}(a,b) & f_{xy}(a,b) \\
                    f_{xy}(a,b) & f_{yy}(a,b)
                   \end{bmatrix} \Bigg |
\]
\begin{tabbing}
  \uline{Teorema}: \= Se $D(a,b) > 0$, então $(a,b)$ é um ponto extremal. \\[5pt]
  \> Se $D(a,b) < 0$, então $(a,b)$ é um ponto de sela.
\end{tabbing}
\begin{tabbing}
  \uline{Teorema}: \= Se $(a,b)$ é um ponto extremal \\[5pt]
  \>\begin{minipage}{\linewidth}
      \begin{itemize}
        \setlength\itemsep{2px}
        \item Se $f_{xx}(a,b) > 0$, então $(a,b)$ é um ponto mínimo local.
        \item Se $f_{xx}(a,b) < 0$, então $(a,b)$ é um ponto máximo local.
      \end{itemize}
    \end{minipage}
\end{tabbing}


\pagebreak


\end{document}
